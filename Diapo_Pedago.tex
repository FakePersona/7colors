\documentclass{beamer}

\usepackage[utf8]{inputenc} 
\usepackage[T1]{fontenc}
\usepackage{lmodern}
\usepackage{graphicx}
\usepackage[french]{babel}

\begin{document}

\begin{frame}
\frametitle{Calcul de la complexité du tri bulle}

Appliquons le tri bulle sur un tableau d'entiers de $\{1, \ldots, n\}$, et comptons le nombre de comparaisons effectuées.
\pause
  \begin{itemize}
    \item Au premier passage, on parcours tous les éléments pour placer $n$ à la fin du tableau. Cela demande $n-1$ comparaisons.
    \item Ensuite, on parcours les autres premiers éléments et on place $n-1$ en avant dernière position. Cela demande $n-2$ comparaisons.
    \item On recommence jusqu'à avoir trié le tableau.
    \item On a alors effectué $(n-1) + (n-2) + \ldots + 1$ comparaisons
  \end{itemize}
\pause
Au total, il faut $\frac{n^2 - n}{2}$ comparaisons pour trier un tableau de taille $n$.

\end{frame}


\begin{frame}
\frametitle{Analyse de complexité du tri bulle}

L'algorithme du tri bulle sur un tableau de taille $n$ effectue $\frac{n^2-n}{2}$ comparaisons.

Lorsque $n$ devient très grand, et en s'affranchissant des constantes multiplicatives, on écrit alors que la complexité est en $n^2$.

\pause
Cela signifie concrètement que pour trier un tableau de taille $2 n$, il faudra quatre fois plus de temps que pour trier un tableau de taille $n$.

\end{frame}


\begin{frame}
\frametitle{Complexité du tri fusion}

Notons $C_n$ le nombre de comparaisons effectués par le tri fusion pour un tableau de taille $n$.

\pause
Cette fois ci, si on trie un tableau de taille $2n$ :
\begin{itemize}
  \item On commence par trier les deux sous tableau de taille $n$. D'où $2 \times C_n$ comparaisons.
  \item Puis on les combines en $2n$ comparaisons.
\end{itemize}
On obtient donc $C_{2n} = 2C_n + 2n$.

On peut en déduire que la complexité est en $n \ln(n)$.

\end{frame}


\begin{frame}
\frametitle{Comparaison des complexités.}
% Il faudrait mettre une figure qui présente la courbe de n ln(n) et de n^2 ensemble sur cette frame.

La complexité du tri bulle dépasse vite celle du tri fusion.

Cette mesure est indépendante de l'ordinateur utilisé.

%lancer la simulation.
\end{frame}

\end{document}