\documentclass[a4paper,11pt]{article}
\usepackage{a4wide}%

\usepackage{fullpage}%
\usepackage[T1]{fontenc}%
\usepackage[utf8]{inputenc}%
\usepackage[main=francais,english]{babel}%

\usepackage{graphicx}%
\usepackage{xspace}%
\usepackage{float}

\usepackage{url} \urlstyle{sf}%
\DeclareUrlCommand\email{\urlstyle{sf}}%

\usepackage{mathpazo}%
\let\bfseriesaux=\bfseries%
\renewcommand{\bfseries}{\sffamily\bfseriesaux}

\newenvironment{keywords}%
{\description\item[Mots-clés.]}%
{\enddescription}


\newenvironment{remarque}%
{\description\item[Remarque.]\sl}%
{\enddescription}

\font\manual=manfnt
\newcommand{\dbend}{{\manual\char127}}

\newenvironment{attention}%
{\description\item[\dbend]\sl}%
{\enddescription}

\usepackage{listings}%

\lstset{%
  basicstyle=\sffamily,%
  columns=fullflexible,%
  language=c,%
  frame=lb,%
  frameround=fftf,%
}%

\lstMakeShortInline{|}

\parskip=0.3\baselineskip

\sloppy

%opening
\title{Arcsys: le jeu des 7 couleurs}
\author{Rémi Hutin \and Rémy Sun}
\date{28 février 2016}


\begin{document}

\maketitle

\begin{abstract}
  
\end{abstract}

\section{Jeu de base}

\subsection{Régles}

\subsection{Fonctions de base}

\subsection{Mise à jour efficace du monde des 7 couleurs}

\section{Quelques intelligences artificielles}

\subsection{Joueur aléatoire}

\subsection{Joueur glouton}

\subsection{Joueur hégémonique}

\subsection{Joueur glouton prévoyant}


\section{Intelligence artificielle hybride}

\subsection{Insuffisance des intelligences artificielles proposées}

\subsection{Intelligence artificielle hybride proposée}




\end{document}
